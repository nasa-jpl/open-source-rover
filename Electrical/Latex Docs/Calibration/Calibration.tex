\documentclass[12pt]{article}
\usepackage[margin=1in]{geometry}
\usepackage{setspace}
\usepackage{graphicx}
\usepackage{subcaption}
\usepackage{amsmath}
\usepackage{color}
\usepackage{hyperref}
\usepackage{multicol}
\usepackage{framed}
\usepackage{xcolor}
\usepackage{wrapfig}
\usepackage{float}
\usepackage{fancyhdr}
\usepackage{verbatim}
\pagestyle{fancy}
\lfoot{\textbf{Open Source Rover Calibration}}
\rfoot{Page \thepage}
\lhead{\textbf{\leftmark}}
\rhead{\textbf{\rightmark}}
\cfoot{}
\renewcommand{\footrulewidth}{1.8pt}
\renewcommand{\headrulewidth}{1.8pt}
\doublespacing
\setlength{\parindent}{1cm}

\begin{document}

\title{Open Source Rover: Calibration}
\author{Author: Eric Junkins}

\makeatletter         
\def\@maketitle{
\begin{center}	
	\makebox[\textwidth][c]{ \includegraphics[width=.8\paperwidth]{"Pictures/calibration".png}}
	{\Huge \bfseries \sffamily \@title }\\[3ex] 
	{\Large \sffamily \@author}\\[3ex] 
	\includegraphics[width=.85\linewidth]{"Pictures/JPL logo".png}
\end{center}}
\makeatother

\maketitle

\noindent {\footnotesize Reference herein to any specific commercial product, process, or service by trade name, trademark, manufacturer, or otherwise, does not constitute or imply its endorsement by the United States Government or the Jet Propulsion Laboratory, California Institute of Technology.}

\noindent {\footnotesize \textcopyright  2018 California Institute of Technology. Government sponsorship acknowledged}

% Introduction
\newpage


\tableofcontents




\section{Calibration}
\label{cal section}
In order for the motors to all work correctly to drive the rover we must calibrate the motors and encoders. We will use a program called Ion Studio to assist with this, which can be found at 
\begin{itemize}
	\item \href{http://www.ionmc.com/downloads}{http://www.ionmc.com/downloads} 
\end{itemize}

\noindent One at a time we will go through and check the encoders, motors, and perform the calibration on each of the motors. As you go to spin the drive motors make sure either the robot is fully hanging such that every drive wheel is not in contact with the ground, or you elevate each drive wheel as you attempt to spin that wheel. 

\subsection{RoboClaw Setup}
\begin{enumerate}
	\item Connect to the RoboClaw via USB-micro USB cable
	\item Open Ion Studio
	\item If necessary update the RoboClaw Firmware
	\begin{figure}[H]
 		\centering
		\includegraphics[width=.55\textwidth]{"Pictures/Ion Studio 1".PNG}
 		\caption{Ion Studio Firmware Update}
	\end{figure}
	
	\item Select Connect Selected Unit in the upper left box
	\item Verify that all the motor control parameters are roughly nominal:
	\begin{figure}[H]
 		\centering
		\includegraphics[width=.55\textwidth]{"Pictures/is4".PNG}
 		\caption{}
	\end{figure}
	\begin{itemize}
		\item \textbf{Temperature1:} ~ 30
		\item \textbf{M1/M2 Amps:} ~ 0.01
		\item \textbf{M1/M2 Encoder:} 0
		\item \textbf{M1/M2 Speed:} 0
		\item \textbf{Main Battery:} Between 11.5-16.7V
		\item \textbf{Logic Battery:} ~5V
		\item \textbf{Model:} 2x7a
	\end{itemize}
	\noindent As you perform the calibration and testing in this doc make sure to monitor these settings and make sure they are changing roughly with how you expect they should, for instance the temperature going much higher than 35 would denote something is going wrong, or if you are pulling a large amount of current you might have a short in the wiring. 
	\item Under the General Settings Tab you'll need to change the following settings for each RoboClaw:
	\begin{figure}[H]
 		\centering
		\includegraphics[width=.55\textwidth]{"Pictures/is2".PNG}
 		\caption{}
	\end{figure}
	\begin{enumerate}
		\item \textbf{Setup}:
		\begin{enumerate}
			\item Control Mode: Packet Serial
			\item Multi-Unit: Check Enable Multi-Unit Mode
		\end{enumerate}

		\item \textbf{Serial}:
		\begin{enumerate}
			\item Set the addressed to 128,129,130,131,132 for each respective RoboClaw
			\item Baudrate: 115200
		\end{enumerate}

		\item \textbf{Battery}:
		\begin{enumerate}
			\item Max Main Batter: 17.5V
			\item Min Main Batter: 11.5V
			\item Max Logic Battery: 5.5V
			\item Min Logic Battery: 4.5V
		\end{enumerate}

		\item \textbf{Motors}:
		\begin{enumerate}
			\item M1 Max Current: 3.0A
			\item M2 Max Current: 3.0 A
		\end{enumerate}

		\item \textbf{I/O}:
		\begin{enumerate}
			\item Encoder 1 Mode: For addressed 128,129,130 Quadrature, for addresses 131 and 132 Absolute
			\item Encoder 2 Mode: For addressed 128,129,130 Quadrature, for addresses 131 and 132 Absolute
		\end{enumerate}
	\end{enumerate}
	\noindent \textbf{After changing these settings make sure to save them! Go to \textit{Device $->$ Save Settings}}

\subsection{Drive Motor Calibration}
	
	\item Perform the following for the \textbf{RoboClaw addresses 128,129, and 130}
	\begin{enumerate}
		\item Go to the PWM settings tab
		\begin{figure}[H]
	 		\centering
			\includegraphics[width=.55\textwidth]{"Pictures/is5".PNG}
	 		\caption{}
		\end{figure}

		\item Under the control pane slowly move the slider bar up for M1. 		
		\begin{figure}[H]
	 		\centering
			\includegraphics[width=.55\textwidth]{"Pictures/is5".PNG}
	 		\caption{}
		\end{figure}
		\noindent Verify that when the forward signal is sent to the motor that the wheel spins in the direction that would move the rover as a whole forward (note this is different clockwise vs counterclockwise based on which wheel you are testing). If the wheel moves backwards with respect to the rover direction than switch the motor leads coming from the motor controller, which are M1A and M1B. 

		\item Now as you drive M1 motor forward (which now corresponds to the rover moving forward) verify that M1 Encoder value increases \footnote{If you get no reading values for the encoder re-check your wiring and connections}. If it is decreasing then you have to switch the A and B channels of the encoder signals. These are the EN1 +/- pins on the header pins. 

		\item Repeat 7b and 7c for M2 motor. 
	\end{enumerate}
	\item If once both motors are spinning the correct direction and encoders respond accordingly when commanded through PWM signal tab move then go to the \textit{Velocity Settings} tab.	
	
	\begin{figure}[H]
 		\centering
		\includegraphics[width=.55\textwidth]{"Pictures/is8".PNG}
 		\caption{}
	\end{figure}
	\noindent Here we'll be tuning the motor settings. This needs to be done to have fluid control of the motor, and needs to be done for each individual motor as they will all differ slightly out of the box, so each will be individually tuned. For more information on motor PID controls and what they do you can visit:
	\begin{itemize}
		\item \href{https://medium.com/pollenrobotics/an-introduction-to-pid-control-with-dc-motor-1fa3b26ec661}{https://medium.com/pollenrobotics/an-introduction-to-pid-control-with-dc-motor-1fa3b26ec661}
		\item \href{https://www.elprocus.com/the-working-of-a-pid-controller/}{https://www.elprocus.com/the-working-of-a-pid-controller/}
	\end{itemize}
	\begin{enumerate}
		\item Begin by entering in the QPPS (quadrature pulses per second), or the maximum number of encoder ticks per second. To figure this out drive the motor at max in the PWM setting and look at the speed.

		\begin{figure}[H]
	 		\centering
			\includegraphics[width=.55\textwidth]{"Pictures/is7".PNG}
	 		\caption{}
		\end{figure}

		\item Now hit the Tune M1 button. This will populate the \textit{Velocity P I D} variables on the side. Repeat this step for M2 motor as well.

		\begin{figure}[H]
	 		\centering
			\includegraphics[width=.55\textwidth]{"Pictures/is9".PNG}
	 		\caption{}
		\end{figure}
	
		\item Once both motors are tuned from this motor controller go to \textit{Device $->$ Save Settings} and then exit. Repeat all of steps 6 and 8 for the other drive motors. 
	\end{enumerate}

\subsection{Corner Motor Calibration}
	\item Perform the following for the \textbf{RoboClaw addresses 131 and 132}
	\begin{enumerate}
		\item Perform step 6 for the Corner motors, making sure to set the encoder type to absolute. 
		\item Go to the \textit{PWM Settings} tab and slowly drive the M1 corner forward. Verify that the encoder value goes up. If it does not switch the motor wire leads at M1A and M1B. \footnote{As opposed to the drive motors the corners use absolute encoders that are defined in one direction only, so you cannot switch the direction of the encoder}
		\item Repeat 9b for M2 motor
		\item You'll might need to set the absolute encoder zero point in a different location. In order to avoid clipping\footnote{Clipping would be where you go past the 360 deg mark on the encoder and it starts again at 0} of the range of the encoder we'll set the mid-point of the absolute encoder to be the middle of the turning arc. 
		\begin{enumerate} 
			\item Navigate to the \textit{PWM Settings} tab to move the corner to its' "zero" position, which is where the wheel is directly forward. 
			\item Unscrew the small gear at the bottom of the absolute encoder
			\item Once the absolute encoder shaft is uncoupled to the robot spin the encoder until you determine the max value. This should be somewhere in the range of 1400-1800 ish\footnote{If you are getting values of close to 2000 and then it is staying above 2000 and not resetting to 0 while spinning it means the Voltage divider board is not working properly. Check to make sure you didn't put the resistors in backwards}.
			\item Spin the encoder until it is reading values of roughly half the max value and reattach the gear, coupling it back to the rover. 
		\end{enumerate}
		
		\item For each motor M1 and M2 get what is now the new min and max encoder values. Using the \textit{PWM Settings} tab move the motors until they run into the physical hard stops. Record the min and max values for each, but for the min values add about 15 and the max subtract about 15\footnote{This is to avoid driving into the hard stop, we'll impose a software 'soft stop'.}
		
		\item Perform the Velocity tuning for each of the corners similar to the drive motors. You will have to estimate the QPPS value for the corners as you cannot spin them at fully speed without running into the physical hard stop very quickly. 

		\item Navigate to the \textit{Position Settings} tab. 

		\begin{figure}[H]
	 		\centering
			\includegraphics[width=.55\textwidth]{"Pictures/is10".PNG}
	 		\caption{}
		\end{figure}
		
		\item Enter in your Min and Max values for each M1 and M2 motor

		\begin{figure}[H]
	 		\centering
			\includegraphics[width=.55\textwidth]{"Pictures/is11".PNG}
	 		\caption{}
		\end{figure}	

		\item Change the Autotune type to \textbf{PID} and then tune the M1 and M2 motors\footnote{We've found this is more successful if you tune them with the wheels down on a smooth surface, but supporting the weight of the robot during tuning}. 

		\begin{figure}[H]
	 		\centering
			\includegraphics[width=.55\textwidth]{"Pictures/is12".PNG}
	 		\caption{}
		\end{figure}	

		\item To test this now using the Controls for M1 and M2 move them to their middle positions (halfway between min and max) and verify that the wheels all should point entirely straight forward now. If they do not re-check the min and max values, and ensure that no signal clipping is happening on the encoders.

		\item Repeat Step 9 for the remaining corner motors
	\end{enumerate}

\end{enumerate} 

\end{document}